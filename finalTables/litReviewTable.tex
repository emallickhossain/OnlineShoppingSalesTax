\begin{table}
\centering
\caption{Estimated Elasticities}
\label{tab:elasticity}
\begin{tabular}{lll}
\hline
\textbf{Elasticity Type}           & \textbf{Paper}                     & \textbf{Estimate}     \\
\hline \hline
\multirow{6}{3.5cm}{\textbf{Cross-border Price-Expenditure}} &
  Asplund, Friberg, and Wilander (2007) -- Foreign price    & 0.2 to 0.5   \\
& Asplund, Friberg, and Wilander (2007) -- Domestic price & -0.2 to -1.3 \\
& Agarwal, Marwell, and McGranahan (2017)                 & -2 to -30    \\
& Davis (2011)                                        & -2.2 to -3.6 \\
& Agarwal, Chomsisengphet, Qian, and Xu (2017)            & -2.3         \\
& Mikesell (1970)                                     & -6.3         \\ \hline
\multirow{5}{3.5cm}{\textbf{Tax-Purchase}} &
  Scanlan (2007)                                        &  0.0          \\
& Ballard and Lee (2007)                               & -0.2          \\
& Alm and Melnik (2005)                                & -0.5          \\
& Einav, Knoepfle, Levin, and Sundaresan (2014)        & -1.8          \\
& Goolsbee (2000)                                     & -2.3          \\ \hline
\multirow{2}{3.5cm}{\textbf{Tax-Quantity}} &
  Anderson, Fong, Simester, and Tucker (2010)          & -1.9 to -2.9  \\
& Ellison \& Ellison (2009)                           & -6            \\ \hline
\multirow{3}{3.5cm}{\textbf{Tax-Expenditure}} &
  Baugh, Ben-David, and Park (2018)                      & -1.2 to -1.4  \\
& Houde, Newberry, and Seim (2017)                     & -1.3          \\
& Hu and Tang (2014)                                  & -3.75 to -4.5 \\ \hline
\end{tabular}
\caption*{Note: Early research focused on how taxes influenced the binary decision of whether or not to make an online purchase. Subsequent research has looked at how sales taxes affect actual online expenditures or quantities purchased. In order to delineate between these, I use "tax-purchase elasticity" to refer to the effect on the purchase decision while "tax-quantity elasticity" and "tax-expenditure elasticity" refer to the effect on online purchase quantities and expenditures, respectively.}
\end{table}
